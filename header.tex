\documentclass[12pt]{article}

% Packages to include in document.
\usepackage[margin=.75in]{geometry}
\usepackage{amsmath}
\usepackage{amssymb}
\usepackage{mathtools}
\usepackage{parskip}
\usepackage{mathrsfs}
\usepackage{setspace}
\usepackage{float}
\usepackage[hidelinks]{hyperref}
\usepackage[noabbrev,capitalize,nameinlink]{cleveref}
\usepackage[compact]{titlesec}
\usepackage{amsthm}
\usepackage[dvipsnames]{xcolor}
\usepackage{multirow}
\usepackage{multicol}
\usepackage{wasysym}
\usepackage{graphicx}
\usepackage{fancyhdr}
\usepackage[b]{esvect}
\usepackage{esint}
\usepackage{letltxmacro}
\usepackage[shortlabels]{enumitem}
\usepackage{tikz-cd}
\usepackage{xparse}
\usepackage{physics}
\usepackage{nccmath}
\usepackage{siunitx}
\usepackage{etoolbox}
\usepackage{listings}
\usepackage{centernot}
\usepackage[useregional]{datetime2}
\usepackage{caption}
\usepackage{stmaryrd}
\usepackage[nottoc,numbib]{tocbibind}

% Reduce spacing around section labels and headers.
\titlespacing{\section}{0pt}{*0}{*0}
\titlespacing{\subsection}{0pt}{*0}{*0}
\titlespacing{\subsubsection}{0pt}{*0}{*0}
\setlength{\headheight}{15pt}
\setlength{\headsep}{12pt}

% Define listing environment format.
\lstset{
	basicstyle=\ttfamily,
	numbers=left,
	stepnumber=1,
	mathescape=true,
	showstringspaces=false,
	tabsize=4,
	breaklines=true,
	breakatwhitespace=false
}

% amsthm definition style, numbering follows sections.
\theoremstyle{definition}
\newtheorem{theorem}{Theorem}%[section]
\newtheorem{definition}{Definition}%[section]
\newtheorem{lemma}{Lemma}%[section]
\newtheorem{corollary}{Corollary}[theorem]

% Math operators to be typeset in roman font.
\DeclareMathOperator{\id}{id}
\DeclareMathOperator{\rng}{rng}
\DeclareMathOperator{\im}{im}
\DeclareMathOperator{\mat}{Mat}
\let\hom\relax
\DeclareMathOperator{\hom}{Hom}
\DeclareMathOperator{\gl}{GL}
\DeclareMathOperator{\codim}{codim}
\DeclareMathOperator{\proj}{proj}
\DeclareMathOperator{\cov}{Cov}
\DeclareMathOperator{\varn}{Var}
\DeclareMathOperator{\inter}{Int}
\DeclareMathOperator{\exter}{Ext}
\DeclareMathOperator{\sgn}{sgn}
\DeclareMathOperator{\divop}{div}
\DeclareMathOperator{\curlop}{curl}
\DeclareMathOperator{\gradop}{grad}
\DeclareMathOperator{\vol}{Vol}
\DeclareMathOperator{\vspan}{span}
\DeclareMathOperator{\rnk}{rank}
\DeclareMathOperator{\aut}{Aut}
\DeclareMathOperator{\supp}{supp}
\DeclareMathOperator{\ord}{ord}

% Prefered symbols for certain items.
\renewcommand{\epsilon}{\varepsilon}
\renewcommand{\emptyset}{\varnothing}
\renewcommand{\vec}{\vv}

% Commonly used commands.
\newcommand{\set}[1]{\ensuremath{\qty{#1}}}
\newcommand{\suchthat}{\, \middle| \,}
\newcommand{\R}{\ensuremath{\mathbb{R}}}
\newcommand{\Q}{\ensuremath{\mathbb{Q}}}
\newcommand{\Z}{\ensuremath{\mathbb{Z}}}
\newcommand{\N}{\ensuremath{\mathbb{N}}}
\newcommand{\C}{\ensuremath{\mathbb{C}}}
\newcommand{\E}{\ensuremath{\mathbb{E}}}
\newcommand{\F}{\ensuremath{\mathbb{F}}}
\newcommand{\solution}{\emph{Solution.} }
\newcommand{\inv}[1]{{#1}^{-1}}
\newcommand{\biImplies}{\ensuremath{\Longleftrightarrow}}
\newcommand{\given}{\,|\,}
\newcommand{\restrict}[2]{\ensuremath{\left.#1\right|_{#2}}}
\newcommand{\argmin}{\ensuremath{\mathop{\arg\min}\limits}}
\newcommand{\argmax}{\ensuremath{\mathop{\arg\max}\limits}}
\newcommand{\nimplies}{\ensuremath{\centernot\implies}}
\newcommand{\multichoose}[2]{\ensuremath{\left(\kern-.3em\left(\genfrac{}{}{0pt}{}{#1}{#2}\right)\kern-.3em\right)}}
\newcommand{\wrapbox}[1]{\fbox{\parbox{\textwidth}{#1}}}
\newcommand{\supto}[1]{\ensuremath{\xrightarrow{\;#1\;}}}

% Change fraction size based on context.
\LetLtxMacro{\oldfrac}{\frac}
\renewcommand{\frac}[2]{
\mathchoice
	{\oldfrac{#1}{#2}}		% display style
	{\mfrac{#1}{#2}}		% text style
	{\oldfrac{#1}{#2}}		% script style
	{\oldfrac{#1}{#2}}		% script-script style
}

% Change binomial coefficient size based on context.
\LetLtxMacro{\oldbinom}{\binom}
\renewcommand{\binom}[2]{
\mathchoice
	{\oldbinom{#1}{#2}}		% display style
	{\mbinom{#1}{#2}}		% text style
	{\oldbinom{#1}{#2}}		% script style
	{\oldbinom{#1}{#2}}		% script-script style
}

% Auto-resizing left and right delimiters.
\DeclarePairedDelimiter{\floor}{\lfloor}{\rfloor}
\DeclarePairedDelimiter{\ceil}{\lceil}{\rceil}
\DeclarePairedDelimiter{\anglebrack}{\langle}{\rangle}
\DeclarePairedDelimiter{\doublebrack}{\llbracket}{\rrbracket}

% Place index labels above and below, not to the right.
\LetLtxMacro{\oldsum}{\sum}
\renewcommand{\sum}{\oldsum\limits}
\LetLtxMacro{\oldlim}{\lim}
\renewcommand{\lim}{\oldlim\limits}
\LetLtxMacro{\oldbigcup}{\bigcup}
\renewcommand{\bigcup}{\oldbigcup\limits}
\LetLtxMacro{\oldbigcap}{\bigcap}
\renewcommand{\bigcap}{\oldbigcap\limits}
\LetLtxMacro{\oldprod}{\prod}
\renewcommand{\prod}{\oldprod\limits}
\LetLtxMacro{\oldsup}{\sup}
\renewcommand{\sup}{\oldsup\limits}
\LetLtxMacro{\oldinf}{\inf}
\renewcommand{\inf}{\oldinf\limits}
\LetLtxMacro{\oldlimsup}{\limsup}
\renewcommand{\limsup}{\oldlimsup\limits}
\LetLtxMacro{\oldliminf}{\liminf}
\renewcommand{\liminf}{\oldliminf\limits}
\LetLtxMacro{\oldbigsqcup}{\bigsqcup}
\renewcommand{\bigsqcup}{\oldbigsqcup\limits}
\LetLtxMacro{\oldbigoplus}{\bigoplus}
\renewcommand{\bigoplus}{\oldbigoplus\limits}
\LetLtxMacro{\oldbigotimes}{\bigotimes}
\renewcommand{\bigotimes}{\oldbigotimes\limits}
\LetLtxMacro{\oldbigwedge}{\bigwedge}
\renewcommand{\bigwedge}{\oldbigwedge\limits}
\LetLtxMacro{\oldbigvee}{\bigvee}
\renewcommand{\bigvee}{\oldbigvee\limits}
\LetLtxMacro{\oldmax}{\max}
\renewcommand{\max}{\oldmax\limits}
\LetLtxMacro{\oldmin}{\min}
\renewcommand{\min}{\oldmin\limits}

% Augmented matrix definition. Usage:
% \begin{(x)matrix}[cc|c]
% 	1 & 2 & 3 \\
%	4 & 5 & 6 \\
% \end{(x)matrix}
% where (x) can be p, b, v, etc.
\makeatletter
\renewcommand*\env@matrix[1][*\c@MaxMatrixCols c]{%
	\hskip -\arraycolsep
	\let\@ifnextchar\new@ifnextchar
	\array{#1}}
\makeatother
